
% -------------------------------------------------------
%  English Abstract
% -------------------------------------------------------


\pagestyle{empty}

\begin{latin}

\begin{center}
\textbf{Abstract}
\end{center}
\baselineskip=.8\baselineskip

With constant increase in the rate of VLSI circuits manufactured in sites separate from the designers and computer architects, global concern regarding the possibility of integration of malware by the manufacturing foundries has arisen. Particularly, one main issue that affects reliability of the chips is modifications or additions with malicious intention, known as Hardware Trojans, which are easily applicable during design and manufacturing phase of chips. There has been an increasing fraud in chip-set manufacturing. Hardware Trojans may leak confidential information outside the chip, to the attacker, may alter the function of circuit, or completely fail a system.

 Hence search for new Trojan Detection methods is absolutely essential. Almost all the present methods are restricted, in that they are suitable only for small Trojans or the gigantic ones. This project strives to fill the gap, by introducing a combined size-aware approach, which is well-suited to striking a balance between tiny and very large Systems-on-Chip. Comparable in speed, our approach is able to offer higher accuracy than its predecessors at the expense of a more complex test design. 


\bigskip\noindent\textbf{Keywords}:
Hardware Trojan Detection, Reliability, System-on-Chip, VLSI

\end{latin}

\newpage
