\chapter{مقدمه}

نخستین فصل این پایان‌نامه به معرفی مسئله، بیان اهمیت موضوع، ادبیات موضوع،
اهداف تحقیق و معرفی ساختار پایان‌نامه می‌پردازد.



\section{تعریف مسئله }

مسئله‌ی \textbf{کشف ترواهای سخت افزاری: }
مدار مجتمع یک مدار الکترونیکی است که بر روی یک تراشه از مواد نیمه‌ها‌‌دی ساخته می‌شود و میلیون‌ها ترانزیستور یا سایر عناصر الکترونیکی می‌تواند روی آن قرار گیرد. این مدارات در تمامی ابزارهای الکترونیکی از رایانه‌ها‌‌ و تلفنهای همراه گرفته تا ابزارهای نظامی و فضایی کاربرد دارند. به طور معمول و به دلیل کاهش هزینه، تولید انبوه این مدارها اکثراً به کارخانه‌های غیر قابل اعتماد سپرده می‌شود. هرگونه اختلاف آگاهانه بین طراحی مدار و سخت افزار تولیدی، که هدف آن تغییر، اختلال در، یا سرقت اطلاعات از عملکرد سخت افزار باشد، به معنی اضافه شدن تروا در مدار است. برای حصول اطمینان درباره‌ی عاری بودن مدار از چنین سخت‌افزارهای بداندیشانه، روش‌های متعددی وجود دارد.  پژوهش با ترکیب دو روشِ آزمون منطقی و اثرات جانبی‌‌، تلاش دارد به یک آزمون با دقت بهتر از کارهای پیشین برسد. هدف دوم این آزمون یافتن نقطه‌ی مرزی برای مقایسه کارایی دو روش مذکور، با در نظر گرفتن اندازه‌ی تروا است.

\section{اهمیت موضوع}

برطبق گزارشهای وزارت دفاع آمریکا$ {\cite{tforce2005}}$
و اسناد وزارت بازرگانی آمریکا $\cite{commerce2010}$ مدارات مجتمع به شدت در برابر فعالیتهای بداندیشانه و مخرب، آسیب پذیر شده‌اند. تروا‌های سخت‌افزاری مهمترین مساله امنیت و قابلیت اطمینان مدارات مجتمع در سالهای اخیر بوده است.


\section{ادبیات موضوع}

تشخیص تروا‌های سخت‌افزاری بسیار سخت است. چرا که اثر آنها بر عملکرد مدار همیشه قابل رؤیت نیست. یک تروا که به صورت حرفه‌ای طراحی شده باشد، ممکن است شامل تعداد اندکی دروازه منطقی باشد که در مکان‌های مختلف مدار درج شده باشند. بنابراین تغییرات در پارامترهای مدار تقریبا قابل چشم‌پوشی است. از طرفی از آنجا که مدار فعال‌کننده تروا معمولاً به نحوی انتخاب می‌شود که در شرایط نادری فعال شود و خروجی مدار تروا تا حد امکان قابل مشاهده نباشد، تشخیص تروا با استفاده از آزمون‌های رایج، بسیار مشکل خواهد بود. از طرفی به  علت طیف گسترده تروا‌ها، ارائه مدلی که تمامی تروا‌ها را بتواند مدل کند و جهت تشخیص تروا به کار رود، ناممکن است.
اخیراً روش‌های مختلفی برای تشخیص تروا ارائه شده‌است. این روش‌ها می‌توانند به سه دسته روش‌های تحلیل اثرات جانبی، روش‌های فعالسازی تروا و معماری‌های نظارتی، دسته‌بندی شوند. در دسته اول پارامترهای جانبی مدار تحلیل می‌شود تا براساس تغییرات آن، حضور تروا تشخیص داده شود. در این روش‌ها ممکن است تحلیل‌های مبتنی بر توان
$\cite{tehranipoor2010survey,agrawal2007trojan,rad2010sensitivity}$
، مبتنی بر جریان
$\cite{wang2008hardware}$
و مبتنی بر تاخیر
$\cite{jin2008hardware,li2008speed}$
انجام شود. این روش‌ها وقتی موثر هستند که اثر تروا بر پارامترهای جانبی شدید و کاملاً قابل تشخیص باشد. اما مسائلی وجود دارد که اثر تروا بر این پارامترها را کم می‌کند. برای مثال نویز اندازه‌گیری، تغییرات فرآیند و تغییرات محیطی می‌تواند اثر تروا بر این پارامترها را کمرنگ کند.
دسته دوم روش‌هایی هستند که قصد فعالسازی تروا به طور کامل را دارند
$ \cite{jha2008randomization,banga2008region,wolff2008towards,salmani2012novel}$
. با فعال شدن تروا، به احتمال بالا با چک کردن خروجی‌ها‌‌ی مدار، عملکرد مخرب آن قابل مشاهده خواهد بود. مساله اصلی در این روش‌ها این است که زمان لازم برای فعالسازی تروا به طور کامل چقدر باشد. گاهی اوقات زمان فعالسازی آنقدر زیاد است که استفاده از این روش به صرفه نخواهد بود. از طرفی یک طراح تروا حرفه‌ای ممکن است تروایی طراحی کند که در شرایط بسیار نادر فعال شود و استفاده از این روش را مشکل سازد. همچنین ممکن است بعضی تروا‌ها اثری بر خروجی مدار نداشته باشند $\cite{wang2008detecting}$.
ساختارهای ناظر برای جلوگیری از آسیبهای ناشی از تروا ارائه شده‌اند. برای نمونه ساختار DEFENSE در طراحی عملیاتی مدارات منطقی تعبیه می‌شود تا به صورت بی درنگ بر امنیت مدار نظارت کند $\cite{abramovici2009integrated}$. در $\cite{kim2009trojan}$ یک معماری گذرگاه در SoC‌ها‌‌ ارائه شده‌است که نسبت به درج تروا مقاوم سازی شده به نحوی که از دسترسی نامطمئن به داده‌های امن جلوگیری کند. مشکل این روش این است که نظارت بر تمام اجزای مداری با میلیون‌ها دروازه منطقی، ناممکن است.
هدف ما در بخش نخست این پژوهش ارائه روش‌های نوینی برای تشخیص تروا است. این روش‌ها باید:
\\
1) برای تشخیص تروا‌هایی که حتی از تعداد کمی دروازه منطقی تشکیل شده‌اند، کارا باشند.
\\
2) اثر تغییرات فرآیند و تغییرات محیطی بر این روش‌ها باید حداقل باشد.
\\
3) اندازه‌گیری‌ها و فرآیند تشخیص در این روش‌ها باید تا حد امکان دارای حداقل سربار زمانی، سربار هزینه و سربار سخت‌افزاری باشد و استفاده از آن به سهولت ممکن باشد.

برای بهبود کارایی روش‌های تشخیص تروا و رفع محدودیت‌های آنها، روش‌های متعددی توسط جامعه محققان اطمینان و امنیت سخت‌افزاری ارائه شده که هدف آنها تغییر روال طراحی کنونی است. به این روش‌ها، روش‌های طراحی برای اطمینان  سخت‌افزاری می گویند $\cite{tehranipoor2010survey}$. هدف از این روش‌های بازدارنده تروا این است که مانع درج تروا  شوند و تشخیص تروا‌ها را تسهیل کنند. برخلاف روش‌های تشخیص تروا که روش‌های منفعلانه هستند، روش‌های طراحی مطمئن، روش‌هایی فعال هستند. یعنی ساختار مدار را به نحوی تغییر می‌دهند تا مانع از درج تروا شوند.
اکثر روش‌های بازدارنده از تروا، با هدف تسهیل در تشخیص تروا با استفاده از روش‌های تحلیل اثرات جانبی ارائه شده‌اند. از این پس به این روش‌ها، روش‌های مبتنی بر اثرانگشت  اثرات جانبی گفته می‌شود. بعضی از این روش‌ها صرفا امکاناتی برای اندازه‌گیری پارامترهای جانبی فراهم می‌کنند ولی برخی دیگر مقادیر اندازه‌گیری شده را با مقادیر آستانه‌ای که از قبل تعریف شده‌اند، مقایسه می‌کنند. سربار طراحی عمده ترین چالش این روش‌هاست. 
در برخی از روش‌های طراحی مطمئن، با این فرض که طراح تروا، مدار تحریک تروا را از قسمت‌هایی انتخاب می‌کند که به ندرت فعال می‌شوند(کنترل‌پذیری پایینی دارند)، هدف، حذف رخدادهای نادر است. روش درج فلیپ فلاپ‌های پویش $\cite{salmani2009new}$ و روش ولتاژ معکوس $\cite{banga2009vitamin}$ با هدف متعادل کردن فرکانس گذار سیگنال‌های داخلی برای حذف رخدادهای نادر، ارائه شده‌اند. از طرفی برخی روش‌ها بدون حذف رخدادهای نادر، کاری می‌کنند که طراح تروا در تشخیص رخدادهای نادر دچار خطا شود. روش مبهم سازی
$\cite{chakraborty2009security}$
به نوعی ساختار واقعی مدار را پنهان می‌کند تا حمله‌کننده نتواند احتمال واقعی رخدادها را حساب کند و براساس آن در انتخاب محل درج تروا به خطا رود. مشکل اکثر این روش‌ها سربار طراحی است و اینکه در برخی موارد فرضیه استفاده از رخدادهای نادر چندان درست نیست. در مقابل روش‌های دیگری هستند که برای اینکه بر چنین فرضیاتی استوار نباشند، از روش طراحی برای آزمون تروا  DFTT\footnote{\lr{Design for Trojan Testability}} که در $\cite{jin2010dftt}$ ارائه شده‌است استفاده می‌کنند. این روش‌ها سربار سخت‌افزاری بیشتری دارند.
برخی دیگر از روش‌ها نیز در دسته روش‌های طراحی مطمئن جای میگیرند. روش‌هایی که هدفشان حفاظت از IP‌ است. در
$\cite{drzevitzky2009proof,love2011enhancing}$
مفهوم سخت‌افزار حامل اثبات PCH\footnote{\lr{Proof Carrying Hardware}}  ارائه شده‌است که مبتنی بر روش حفاظت نرم‌افزاری کد حامل اثبات PCC\footnote{\lr{Proof Carrying Code}} است. این روش برای ممانعت از درج تروا در IP‌ ارائه شده‌است.
هدف ما در بخش دوم این پژوهش این است که روش‌های جدیدی برای تسهیل روش‌های تشخیص تروای ارائه شده و یا برای مقاوم سازی مدار در برابر درج تروا ارائه کنیم.. این روش‌ها باید:
\\
1) تا حد امکان سربار زمانی و سخت‌افزاری کمتری نسبت به روش‌های پیشین داشته باشند.
\\
2) طراحی مطمئن مدار را یا بسیار راحت کنند و یا ابزاری برای خودکار سازی این روند ارائه کنند. 




\section{اهداف تحقیق}

در این پایان‌نامه سعی می‌شود که مسئله‌ی کشف ترواهای سخت‌افزاری مورد مطالعه قرار گیرد.برای حصول اطمینان درباره‌ی عاری بودن مدار از چنین سخت‌افزارهای بداندیشانه، روش‌های متعددی وجود دارد. این پژوهش با ترکیب و تمرکز رویِ دو روشِ آزمون منطقی و اثرات جانبی‌‌، تلاش دارد در وهله نخست به یک آزمون کشف تروا با دقت بهتر از کارهای پیشین برسد. هدف دوم این آزمون یافتن نقطه‌ی مرزی برای مقایسه کارایی دو روش مذکور، با در نظر گرفتن اندازه‌ی تروا است. بعد از مطالعه‌ی کارهای انجام شده در این زمینه سعی می‌شود که مسئله به صورت دقیق‌تر مورد بررسی قرار گیرد.

\section{ساختار پایان‌نامه}

این پایان‌نامه شامل پنج فصل است. 
فصل دوم دربرگیرنده‌ی تعاریف اولیه‌ی مرتبط با پایان‌نامه است. 
در فصل سوم مسئله‌ی کشف تروا و کارهای مرتبطی که در این زمینه انجام شده به تفصیل بیان می‌گردد. 
در فصل چهارم نتایج جدیدی که در این پایان‌نامه به دست آمده ارائه می‌گردد. در این فصل، به صورت دقیق و گام به گام الگوریتم‌ها، برنامه‌ها و شبیه سازهایی که در این پژوهش تولید و یا استفاده شده‌اند معرفی و بررسی می‌شوند. در نهایت خروجی شبیه‌سازی‌ها ارائه خواهد شد.
فصل پنجم به نتیجه‌گیری و پیش‌نهادهایی برای کارهای آتی خواهد پرداخت.

