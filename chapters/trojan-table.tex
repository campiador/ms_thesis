
\begin{table}
	\caption{مقایسه روش‌های مقابله با تروا}
	\label{tcomparison}
	\begin{tabular}{|p{1cm}||p{3cm}|p{3cm}|p{3cm}|p{3cm}|}
		\hline

		نوع روش & نام روش  & تروا‌های قابل ئشخیص & سربار& معایب دیگر \\ \hline \hline
		روش ‌های آزمون منطقی& روش تولید بردارهای آزمون&اغلب تروا‌های با تحریک ترکیبی، تعداد معدودی از تروا‌های با تحریک ترتیبی(  فقط تروا‌هایی که با شرایط خاص فعال می‌شوند)&این روش در حین آزمون انجام می‌شود و سربار زمانی بسیار زیادی دارد.(44.5 ساعت برای مدار c7552 )&طبق نتایج مقاله، پوشش دهی تروا‌ها برای مدارهای ترتیبی نسبت به روش بردارهای تصادفی تفاوت چندانی نداشته و در برخی موارد کمتر شده‌است. ضمن اینکه  این روش نیاز به تحلیل قبلی مدار برای یافتن رخدادهای نادر دارد.
		
		MERO [27]
		
		\\ \cline{2-5}
		&روش شفاف کردن ماژول­ها (افزایش کنترل پذیری و مشاهده پذیری) [29]  &فقط تروا‌های خیلی بزرگ ( فقط تروا‌هایی که در شرایط خاص فعال می‌شوند)&طبق گزارش مقاله 5\% سربار مساحت، 13\% سربار توان مصرفی، 5\% سربار تاخیر و 9 پایه اضافه لازم است. & نحوه اعمال کلیدها به ماژول­ها و ایجاد امضای خروجی، در نتیجه این روش موثر است که در این مقاله توضیح چندانی داده نشده‌است.
		\\ \hline \hline 
	\end{tabular}
\end{table}
\newpage
\begin{table}
	\begin{tabular}{|p{1cm}||p{3cm}|p{3cm}|p{3cm}|p{3cm}|}
		روش ‌های تحلیل اثرات جانبی با رویکرد اندازه ‌گیری توان مصرفی&روش ناحیه بندی [10]  &فقط تروا‌های با تحریک ترتیبی که در شرایط خاص فعال می‌شوند &سربار زمانی حین آزمون زیاد است و هرچه مدار بزرگتر شود، این سربار بیشتر می‌شود. &میزان نتیجه گیری کاملا وابسته به شخص است. چرا که هیچ روند خودکار­سازی برای ناحیه­بندی و ایجاد بردارهای آزمون ارائه نشده‌است. 
		%	\end{tabular}

		
		%\begin{tabular}{|p{1cm}||p{3cm}|p{3cm}|p{3cm}|p{3cm}|}

		\\ \cline{2-5}
		&روش اعمال بردارهای پایدار شده [24]  &اغلب تروا‌های با تحریک ترکیبی، تعداد معدودی از تروا‌های با تحریک ترتیبی ( فقط تروا‌هایی که با شرایط خاص فعال می‌شوند)  &سربار زمانی زیاد در حین آزمون به علت ثابت نگهداشتن بردار ورودی برای مدت زمان تعیین شده لازم ‌است. &افزایش حرارت و در پی آن کاهش عمر مدار به علت افزایش خودخواسته توان مصرفی
		
		\\ \cline{2-5}
		& روش محاسبه جریان از طریق پایه­های تغذیه مختلف [6] &این روش فقط برای یک مدار ترکیبی آزموده شده‌است. برای مدارات ترتیبی خیلی مناسب نخواهد بود. & سربار ابزارهای جانبی اندازه‌گیری دقیق جریان - سربار زمانی حین آزمون&پوشش تروا پایین در حد 50\% برای تروا‌های فعال و 30\% برای تروا‌های غیرفعال 
		\\ \cline{2-5}
		& استفاده از ترتیب دهی مجدد فلیپ فلاپ‌های پویش [16] &انواع تروا‌های کوچک و بزرگ (فقط تروا‌هایی که با شرایط خاص فعال می‌شوند)  &سربار زمانی حین آزمون & عدم تشخیص برخی تروا‌های با تحریک ترتیبی- نیاز به پایه­های اضافه برای تراشه، بسته به کوچکترین تروا قابل تشخیص
		\\ \hline \hline
		
		روش ‌های تحلیل اثرات جانبی با رویکرد اندازه ‌گیری تاخیر& محاسبه تاخیر مسیرها [7] & فقط تروا‌های با تحریک ترکیبی- تشخیص تروا‌های بزرگ راحت تر است.& سربار زمانی حین آزمون&عدم تشخیص تروا‌هایی که در مسیر بین ورودی و خروجی مدار واقع نشده و در مسیرهای داخلی اند.
		\\ \cline{2-5}
		%\end{tabular}

		
		
		%\begin{tabular}{|p{1cm}||p{3cm}|p{3cm}|p{3cm}|p{3cm}|}

		&محاسبه تاخیر مسیرهای داخلی با استفاده از ثبات‌های سایه [8]  &انواع تروا‌ها با اولویت تروا‌های بزرگ (هرچه اندازه تروا‌ها کوچکتر باشد، نیاز به سربار بیشتر زمانی و پیچیدگی بیشتر طراحی است) &سربار زمانی زیاد حین آزمون- سربار مساحت و توان ناشی از تولید کننده سیگنال کلاک – سربار مساحت و توان ثبات‌های سایه &دقت تشخیص تروا وابسته به گام تغییر فاز کلاک است که پیچیدگی را زیاد می­کند. با افزایش ابعاد مدار شبکه توزیع کلاک دردسرساز می‌شود.
		\\ \hline \hline
		&تشخیص تغییرات تاخیر با استفاده از نوسانگرهای حلقوی [30,31]&فقط تروا‌های با تحریک ترکیبی- تشخیص تروا‌های بزرگ راحت تر است.&سربار مساحت و توان نوسانگرها یا مدارات اضافه جهت ساختن آنها در مدار-  سربار زمانی حین آزمون برای خواندن نتایج- سربار مساحت و توان اندازه گیرهای فرکانس&تعداد نوسانگرها و موقعیت درج آنها کاملا وابسته به دانش فرد طراح است.
		\\ \hline
	\end{tabular}
	
\end{table}
